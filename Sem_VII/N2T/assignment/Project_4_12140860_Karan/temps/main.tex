%%%%%%%%%%%%%%%%%%%%%%%%%%%%%%%%%%%%%%%%%%%%%%%%%%%%%%%%%%%%%%%%%%
%%%%%%%%%%%%%%%%%%%%%%%%%%%%%%%%%%%%%%%%%%%%%%%%%%%%%%%%%%%%%%%%%%
% \setmathfont{TeX Gyre Termes Math}
%Packages
\documentclass[10pt, a4paper]{article}
\usepackage[top=3cm, bottom=4cm, left=2cm, right=2cm]{geometry}
\usepackage{amsmath,amsthm,amsfonts,amssymb,amscd, fancyhdr, color, comment, graphicx, environ}
\usepackage{float}
\usepackage{booktabs}
\usepackage{pifont}
\usepackage{mathrsfs}
\usepackage[math-style=ISO]{unicode-math}
\usepackage{lastpage}
\usepackage[dvipsnames]{xcolor}
\usepackage[framemethod=TikZ]{mdframed}
\usepackage{enumerate}
\usepackage[shortlabels]{enumitem}
\usepackage{fancyhdr}
\usepackage{indentfirst}
\usepackage{listings}
\usepackage{sectsty}
\usepackage{thmtools}
\usepackage{shadethm}
\usepackage{hyperref}
\usepackage{setspace}
\usepackage{adjustbox}
\hypersetup{
    colorlinks=true,
    linkcolor=blue,
    filecolor=magenta,      
    urlcolor=blue,
}
\usepackage{xcolor,colortbl}
%%%%%%%%%%%%%%%%%%%%%%%%%%%%%%%%%%%%%%%%%%%%%%%%%%%%%%%%%%%%%%%%%%
%%%%%%%%%%%%%%%%%%%%%%%%%%%%%%%%%%%%%%%%%%%%%%%%%%%%%%%%%%%%%%%%%%
%Environment setup
\mdfsetup{skipabove=\topskip,skipbelow=\topskip}
\newrobustcmd\ExampleText{%
An \textit{inhomogeneous linear} differential equation has the form
\begin{align}
L[v ] = f,
\end{align}
where $L$ is a linear differential operator, $v$ is the dependent
variable, and $f$ is a given non−zero function of the independent
variables alone.
}
\mdfdefinestyle{theoremstyle}{%
linecolor=black,linewidth=1pt,%
frametitlerule=true,%
frametitlebackgroundcolor=gray!20,
innertopmargin=\topskip,
}
\mdtheorem[style=theoremstyle]{Problem}{Question Number}
\setcounter{Problem}{-1}
\newenvironment{Solution}{\textbf{Solution.}}

\definecolor{codegreen}{rgb}{0,0.6,0}
\definecolor{codegray}{rgb}{0.5,0.5,0.5}
\definecolor{codepurple}{rgb}{0.58,0,0.82}
\definecolor{backcolour}{rgb}{0.95,0.95,0.92}

\lstdefinestyle{mystyle}{
    backgroundcolor=\color{backcolour},   
    commentstyle=\color{codegreen},
    keywordstyle=\color{magenta},
    numberstyle=\tiny\color{codegray},
    stringstyle=\color{codepurple},
    basicstyle=\ttfamily\footnotesize,
    breakatwhitespace=false,         
    breaklines=true,                 
    captionpos=b,                    
    keepspaces=true,                 
    numbers=left,                    
    numbersep=5pt,                  
    showspaces=false,                
    showstringspaces=false,
    showtabs=false,                  
    tabsize=2
}

\lstset{style=mystyle}
%%%%%%%%%%%%%%%%%%%%%%%%%%%%%%%%%%%%%%%%%%%%%%%%%%%%%%%%%%%%%%%%%%
%%%%%%%%%%%%%%%%%%%%%%%%%%%%%%%%%%%%%%%%%%%%%%%%%%%%%%%%%%%%%%%%%%
%Fill in the appropriate information below
\newcommand{\norm}[1]{\left\lVert#1\right\rVert}     
\newcommand\course{XXXX0000}                            % <-- course name   
\newcommand\hwnumber{0}                                 % <-- homework number
\newcommand\Information{Someone}                        % <-- personal information
%%%%%%%%%%%%%%%%%%%%%%%%%%%%%%%%%%%%%%%%%%%%%%%%%%%%%%%%%%%%%%%%%%
%%%%%%%%%%%%%%%%%%%%%%%%%%%%%%%%%%%%%%%%%%%%%%%%%%%%%%%%%%%%%%%%%%
%Page setup
\pagestyle{fancy}
\headheight 35pt
\lhead{12140860 \hspace*{4cm} Karan Sunil Kumbhar}
\rhead{\includegraphics[width=1.2cm]{./images/logo.png}}
\lfoot{}
\pagenumbering{arabic}
\cfoot{\small\thepage}
\rfoot{}
\headsep 1.2em
\renewcommand{\baselinestretch}{1.25}
%%%%%%%%%%%%%%%%%%%%%%%%%%%%%%%%%%%%%%%%%%%%%%%%%%%%%%%%%%%%%%%%%%
%%%%%%%%%%%%%%%%%%%%%%%%%%%%%%%%%%%%%%%%%%%%%%%%%%%%%%%%%%%%%%%%%%
%Add new commands here
\renewcommand{\labelenumi}{\alph{enumi})}
\newcommand{\Z}{\mathbb Z}
\newcommand{\R}{\mathbb R}
\newcommand{\Q}{\mathbb Q}
\newcommand{\NN}{\mathbb N}
\newcommand{\PP}{\mathbb P}
\DeclareMathOperator{\Mod}{Mod} 
\renewcommand\lstlistingname{Algorithm}
\renewcommand\lstlistlistingname{Algorithms}
\def\lstlistingautorefname{Alg.}
\newtheorem*{theorem}{Theorem}
\newtheorem*{lemma}{Lemma}
\newtheorem{case}{Case}
\newcommand{\assign}{:=}
\newcommand{\infixiff}{\text{ iff }}
\newcommand{\nobracket}{}
\newcommand{\backassign}{=:}
\newcommand{\tmmathbf}[1]{\ensuremath{\boldsymbol{#1}}}
\newcommand{\tmop}[1]{\ensuremath{\operatorname{#1}}}
\newcommand{\tmtextbf}[1]{\text{{\bfseries{#1}}}}
\newcommand{\tmtextit}[1]{\text{{\itshape{#1}}}}

\newenvironment{itemizedot}{
    \begin{itemize} 
    \renewcommand{\labelitemi}{$\bullet$}
    \renewcommand{\labelitemii}{$\bullet$}
    \renewcommand{\labelitemiii}{$\bullet$}
    \renewcommand{\labelitemiv}{$\bullet$}}
    {\end{itemize}}

\catcode`\<=\active\def<{
\fontencoding{T1}\selectfont\symbol{60}\fontencoding{\encodingdefault}}
\catcode`\>=\active\def>{
\fontencoding{T1}\selectfont\symbol{62}\fontencoding{\encodingdefault}}
\catcode`\<=\active\def<{
\fontencoding{T1}\selectfont\symbol{60}\fontencoding{\encodingdefault}}

%%%%%%%%%%%%%%%%%%%%%%%%%%%%%%%%%%%%%%%%%%%%%%%%%%%%%%%%%%%%%%%%%%
%%%%%%%%%%%%%%%%%%%%%%%%%%%%%%%%%%%%%%%%%%%%%%%%%%%%%%%%%%%%%%%%%%
%Begin now!



\begin{document}
%%%%%%%%%%%%%%%%%%%%%%%%%%%%%%%%%%%%%%%%%%%%%%%%%%%%%%%%%%%%%%%%%%
%%%%%%%%%%%%%%%%%%%%%%%%%%%%%%%%%%%%%%%%%%%%%%%%%%%%%%%%%%%%%%%%%%
%Start the assignment now
%%%%%%%%%%%%%%%%%%%%%%%%%%%%%%%%%%%%%%%%%%%%%%%%%%%%%%%%%%%%%%%%%%
%New problem
\newpage
\begin{Problem}
    Write a Hack assembly program that \textbf{copies} the value from memory location 0 into memory location 2.
\end{Problem}

\begin{Solution}

    \begin{itemize}
        \item \textbf{Before Runnig program}
              \begin{center}
                  \includegraphics*[scale=0.23]{./images/q0_1.png}
              \end{center}

        \item \textbf{After Runnig program}
              \begin{center}
                  \includegraphics*[scale=0.23]{./images/q0_2.png}
              \end{center}

    \end{itemize}

\end{Solution}

%%%%%%%%%%%%%%%%%%%%%%%%%%%%%%%%%%%%%%%%%%%%%%%%%%%%%%%%%%%%%%%%%%


%%%%%%%%%%%%%%%%%%%%%%%%%%%%%%%%%%%%%%%%%%%%%%%%%%%%%%%%%%%%%%%%%%
%New problem
\newpage
\begin{Problem}
    Write a Hack assembly program that \textbf{subtracts} the values stored in memory locations 1 and 2, and stores the
    result in memory location 0.
\end{Problem}

\begin{Solution}

    \begin{itemize}
        \item \textbf{Before Runnig program}
              \begin{center}
                  \includegraphics*[scale=0.23]{./images/q1_1.png}
              \end{center}

        \item \textbf{After Runnig program}
              \begin{center}
                  \includegraphics*[scale=0.23]{./images/q1_2.png}
              \end{center}
    \end{itemize}

    \textbf{Note}

    Here I am subtracting $Ram[0] = Ram[1]-Ram[2]$

\end{Solution}

%%%%%%%%%%%%%%%%%%%%%%%%%%%%%%%%%%%%%%%%%%%%%%%%%%%%%%%%%%%%%%%%%%


%%%%%%%%%%%%%%%%%%%%%%%%%%%%%%%%%%%%%%%%%%%%%%%%%%%%%%%%%%%%%%%%%%
%New problem
\newpage
\begin{Problem}
    Write a Hack assembly program that \textbf{swaps} the values stored in memory locations 0 and 1.
\end{Problem}

\begin{Solution}

    \begin{itemize}
        \item \textbf{Before Runnig program}
              \begin{center}
                  \includegraphics*[scale=0.23]{./images/q2_1.png}
              \end{center}

        \item \textbf{After Runnig program}
              \begin{center}
                  \includegraphics*[scale=0.23]{./images/q2_2.png}
              \end{center}
    \end{itemize}
\end{Solution}

%%%%%%%%%%%%%%%%%%%%%%%%%%%%%%%%%%%%%%%%%%%%%%%%%%%%%%%%%%%%%%%%%%


%%%%%%%%%%%%%%%%%%%%%%%%%%%%%%%%%%%%%%%%%%%%%%%%%%%%%%%%%%%%%%%%%%
%New problem
\newpage
\begin{Problem}
    Write a Hack assembly program that \textbf{checks} if the value stored in memory location 0 equals that stored in
    memory location 1. If they are equal, store 1 in memory location 2; otherwise, store 0.
\end{Problem}

\begin{Solution}

    \begin{itemize}
        \item For \textbf{Not Equal} Numbers
              \begin{itemize}
                  \item \textbf{Before Runnig program}
                        \begin{center}
                            \includegraphics*[scale=0.23]{./images/q3_1_1.png}
                        \end{center}

                  \item \textbf{After Runnig program}
                        \begin{center}
                            \includegraphics*[scale=0.23]{./images/q3_1_2.png}
                        \end{center}
              \end{itemize}
              \pagebreak
        \item For \textbf{Equal} Numbers
              \begin{itemize}
                  \item \textbf{Before Runnig program}
                        \begin{center}
                            \includegraphics*[scale=0.23]{./images/q3_2_1.png}
                        \end{center}

                  \item \textbf{After Runnig program}
                        \begin{center}
                            \includegraphics*[scale=0.23]{./images/q3_2_2.png}
                        \end{center}
              \end{itemize}
    \end{itemize}
\end{Solution}

%%%%%%%%%%%%%%%%%%%%%%%%%%%%%%%%%%%%%%%%%%%%%%%%%%%%%%%%%%%%%%%%%%


%%%%%%%%%%%%%%%%%%%%%%%%%%%%%%%%%%%%%%%%%%%%%%%%%%%%%%%%%%%%%%%%%%
%New problem
\newpage
\begin{Problem}
    Write a Hack assembly program that implements a simple \textbf{loop} to increment the value in memory location 0
    by 1 a total of 5 times, storing the result in memory location 1.
\end{Problem}

\begin{Solution}

    \begin{itemize}
        \item \textbf{Before Runnig program}
              \begin{center}
                  \includegraphics*[scale=0.23]{./images/q4_1.png}
              \end{center}

        \item \textbf{After Runnig program}
              \begin{center}
                  \includegraphics*[scale=0.23]{./images/q4_2.png}
              \end{center}
    \end{itemize}
\end{Solution}

%%%%%%%%%%%%%%%%%%%%%%%%%%%%%%%%%%%%%%%%%%%%%%%%%%%%%%%%%%%%%%%%%%


%%%%%%%%%%%%%%%%%%%%%%%%%%%%%%%%%%%%%%%%%%%%%%%%%%%%%%%%%%%%%%%%%%
%New problem
\newpage
\begin{Problem}

    Write a Hack assembly program that \textbf{reads from the keyboard} and stores the code of the first key at RAM[0]
    and code of the second key at RAM[1] and then adds the codes and stores at RAM[2]. after that it \textbf{blackens}
    the first 16 pixels of row 6 of the screen.
\end{Problem}
\begin{Solution}

    \begin{itemize}
        \item \textbf{Before Runnig program}
              \begin{center}
                  \includegraphics*[scale=0.23]{./images/q5_1.png}
              \end{center}

        \item \textbf{After 1 Key Press}
              \begin{center}
                  \includegraphics*[scale=0.23]{./images/q5_2.png}
              \end{center}
              \pagebreak
        \item \textbf{After 2 Key Press}
              \begin{center}
                  \includegraphics*[scale=0.23]{./images/q5_3.png}
              \end{center}
    \end{itemize}
\end{Solution}

%%%%%%%%%%%%%%%%%%%%%%%%%%%%%%%%%%%%%%%%%%%%%%%%%%%%%%%%%%%%%%%%%%


%%%%%%%%%%%%%%%%%%%%%%%%%%%%%%%%%%%%%%%%%%%%%%%%%%%%%%%%%%%%%%%%%%
%New problem
\newpage
\begin{Problem}
    Write a Hack assembly program that continuously \textbf{checks for keyboard input}. Whenever any key is pressed,
    the program should \textbf{black} the first 16 pixels of the top row(top left corner) of the screen. The program should
    keep running, waiting for additional key presses, and each key press should result in a black line being drawn
    on the screen.
\end{Problem}
\begin{Solution}

    \begin{itemize}
        \item \textbf{Before Runnig program}
              \begin{center}
                  \includegraphics*[scale=0.23]{./images/q6_1.png}
              \end{center}

        \item \textbf{After 10 Key Press}
              \begin{center}
                  \includegraphics*[scale=0.23]{./images/q6_10.png}
              \end{center}
              \pagebreak
        \item \textbf{After 25 Key Press}
              \begin{center}
                  \includegraphics*[scale=0.23]{./images/q6_25.png}
              \end{center}
    \end{itemize}
\end{Solution}

%%%%%%%%%%%%%%%%%%%%%%%%%%%%%%%%%%%%%%%%%%%%%%%%%%%%%%%%%%%%%%%%%%


%%%%%%%%%%%%%%%%%%%%%%%%%%%%%%%%%%%%%%%%%%%%%%%%%%%%%%%%%%%%%%%%%%
%New problem
\newpage
\begin{Problem}
    \textbf{Mult}
\end{Problem}
\begin{Solution}
    \begin{itemize}
        \item \textbf{Before Runnig program}
              \begin{center}
                  \includegraphics*[scale=0.23]{./images/q7_1.png}
              \end{center}

        \item \textbf{After running program using mult.tst file}
              \begin{center}
                  \includegraphics*[scale=0.23]{./images/q7_2.png}
              \end{center}
    \end{itemize}
\end{Solution}

%%%%%%%%%%%%%%%%%%%%%%%%%%%%%%%%%%%%%%%%%%%%%%%%%%%%%%%%%%%%%%%%%%

%%%%%%%%%%%%%%%%%%%%%%%%%%%%%%%%%%%%%%%%%%%%%%%%%%%%%%%%%%%%%%%%%%
%New problem
\newpage
\begin{Problem}
    \textbf{Fill}
\end{Problem}
\begin{Solution}
    \begin{itemize}
        \item \textbf{Before Runnig program}
              \begin{center}
                  \includegraphics*[scale=0.23]{./images/q8_1.png}
              \end{center}

        \item \textbf{After running program using FillAutomatic.tst file}
              \begin{center}
                  \includegraphics*[scale=0.23]{./images/q8_2.png}
              \end{center}
    \end{itemize}
    \pagebreak

    \begin{itemize}
        \item \textbf{When no key pressed}
              \begin{center}
                  \includegraphics*[scale=0.23]{./images/q8_3.png}
              \end{center}

        \item \textbf{When key get pressed}
              \begin{center}
                  \includegraphics*[scale=0.23]{./images/q8_4.png}
              \end{center}
    \end{itemize}
\end{Solution}

%%%%%%%%%%%%%%%%%%%%%%%%%%%%%%%%%%%%%%%%%%%%%%%%%%%%%%%%%%%%%%%%%%
\end{document}


%%%%%%%%%%%%%%%%%%%%%%%%%%%%%%%%%%%%%%%%%%%%%%%%%%%%%%%%%%%%%%%%%%
%%%%%%%%%%%%%%%%%%%%%%%%%%%%%%%%%%%%%%%%%%%%%%%%%%%%%%%%%%%%%%%%%%
