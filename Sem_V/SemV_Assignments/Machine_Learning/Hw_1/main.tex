%%%%%%%%%%%%%%%%%%%%%%%%%%%%%%%%%%%%%%%%%%%%%%%%%%%%%%%%%%%%%%%%%%
%%%%%%%%%%%%%%%%%%%%%%%%%%%%%%%%%%%%%%%%%%%%%%%%%%%%%%%%%%%%%%%%%%
%Packages
\documentclass[10pt, a4paper]{article}
\usepackage[top=3cm, bottom=4cm, left=3.5cm, right=3.5cm]{geometry}
\usepackage{amsmath,amsthm,amsfonts,amssymb,amscd, fancyhdr, color, comment, graphicx, environ}
\usepackage{float}
\usepackage{mathrsfs}
\usepackage[math-style=ISO]{unicode-math}
\setmathfont{TeX Gyre Termes Math}
\usepackage{lastpage}
\usepackage[dvipsnames]{xcolor}
\usepackage[framemethod=TikZ]{mdframed}
\usepackage{enumerate}
\usepackage[shortlabels]{enumitem}
\usepackage{fancyhdr}
\usepackage{indentfirst}
\usepackage{listings}
\usepackage{sectsty}
\usepackage{thmtools}
\usepackage{shadethm}
\usepackage{hyperref}
\usepackage{setspace}
\usepackage{pgfplots}
\hypersetup{
    colorlinks=true,
    linkcolor=blue,
    filecolor=magenta,      
    urlcolor=blue,
}
%%%%%%%%%%%%%%%%%%%%%%%%%%%%%%%%%%%%%%%%%%%%%%%%%%%%%%%%%%%%%%%%%%
%%%%%%%%%%%%%%%%%%%%%%%%%%%%%%%%%%%%%%%%%%%%%%%%%%%%%%%%%%%%%%%%%%
%Environment setup
\mdfsetup{skipabove=\topskip,skipbelow=\topskip}
\newrobustcmd\ExampleText{%
An \textit{inhomogeneous linear} differential equation has the form
\begin{align}
L[v ] = f,
\end{align}
where $L$ is a linear differential operator, $v$ is the dependent
variable, and $f$ is a given non−zero function of the independent
variables alone.
}
\mdfdefinestyle{theoremstyle}{%
linecolor=black,linewidth=1pt,%
frametitlerule=true,%
frametitlebackgroundcolor=gray!20,
innertopmargin=\topskip,
}
\mdtheorem[style=theoremstyle]{Problem}{Problem}
\newenvironment{Solution}{\textbf{Solution.}}

\definecolor{codegreen}{rgb}{0,0.6,0}
\definecolor{codegray}{rgb}{0.5,0.5,0.5}
\definecolor{codepurple}{rgb}{0.58,0,0.82}
\definecolor{backcolour}{rgb}{0.95,0.95,0.92}

\lstdefinestyle{mystyle}{
    backgroundcolor=\color{backcolour},   
    commentstyle=\color{codegreen},
    keywordstyle=\color{magenta},
    numberstyle=\tiny\color{codegray},
    stringstyle=\color{codepurple},
    basicstyle=\ttfamily\footnotesize,
    breakatwhitespace=false,         
    breaklines=true,                 
    captionpos=b,                    
    keepspaces=true,                 
    numbers=left,                    
    numbersep=5pt,                  
    showspaces=false,                
    showstringspaces=false,
    showtabs=false,                  
    tabsize=2
}

\lstset{style=mystyle}
%%%%%%%%%%%%%%%%%%%%%%%%%%%%%%%%%%%%%%%%%%%%%%%%%%%%%%%%%%%%%%%%%%
%%%%%%%%%%%%%%%%%%%%%%%%%%%%%%%%%%%%%%%%%%%%%%%%%%%%%%%%%%%%%%%%%%
%Fill in the appropriate information below
\newcommand{\norm}[1]{\left\lVert#1\right\rVert}     
\newcommand\course{CS550 Machine Learning}                            % <-- course name   
\newcommand\hwnumber{ 1}                                 % <-- homework number
\newcommand\Information{Karan Sunil Kumbhar }                        % <-- personal information
\newcommand\Informatio{Id. - 12140860}
\newcommand\Informati{BTech CSE}
\newcommand\Informat{2025}
%%%%%%%%%%%%%%%%%%%%%%%%%%%%%%%%%%%%%%%%%%%%%%%%%%%%%%%%%%%%%%%%%%
%%%%%%%%%%%%%%%%%%%%%%%%%%%%%%%%%%%%%%%%%%%%%%%%%%%%%%%%%%%%%%%%%%
%Page setup
\pagestyle{fancy}
\headheight 35pt
\lhead{\today}
\rhead{\includegraphics[width=1.5cm]{iitbh.png}}
\lfoot{}
\pagenumbering{arabic}
\cfoot{\small\thepage}
\rfoot{}
\headsep 1.2em
\renewcommand{\baselinestretch}{1.25}
%%%%%%%%%%%%%%%%%%%%%%%%%%%%%%%%%%%%%%%%%%%%%%%%%%%%%%%%%%%%%%%%%%
%%%%%%%%%%%%%%%%%%%%%%%%%%%%%%%%%%%%%%%%%%%%%%%%%%%%%%%%%%%%%%%%%%
%Add new commands here
\renewcommand{\labelenumi}{\alph{enumi})}
\newcommand{\Z}{\mathbb Z}
\newcommand{\R}{\mathbb R}
\newcommand{\Q}{\mathbb Q}
\newcommand{\NN}{\mathbb N}
\newcommand{\PP}{\mathbb P}
\DeclareMathOperator{\Mod}{Mod} 
\renewcommand\lstlistingname{Algorithm}
\renewcommand\lstlistlistingname{Algorithms}
\def\lstlistingautorefname{Alg.}
\newtheorem*{theorem}{Theorem}
\newtheorem*{lemma}{Lemma}
\newtheorem{case}{Case}
\newcommand{\assign}{:=}
\newcommand{\infixiff}{\text{ iff }}
\newcommand{\nobracket}{}
\newcommand{\backassign}{=:}
\newcommand{\tmmathbf}[1]{\ensuremath{\boldsymbol{#1}}}
\newcommand{\tmop}[1]{\ensuremath{\operatorname{#1}}}
\newcommand{\tmtextbf}[1]{\text{{\bfseries{#1}}}}
\newcommand{\tmtextit}[1]{\text{{\itshape{#1}}}}

\newenvironment{itemizedot}{\begin{itemize} \renewcommand{\labelitemi}{$\bullet$}\renewcommand{\labelitemii}{$\bullet$}\renewcommand{\labelitemiii}{$\bullet$}\renewcommand{\labelitemiv}{$\bullet$}}{\end{itemize}}
\catcode`\<=\active \def<{
\fontencoding{T1}\selectfont\symbol{60}\fontencoding{\encodingdefault}}
\catcode`\>=\active \def>{
\fontencoding{T1}\selectfont\symbol{62}\fontencoding{\encodingdefault}}
\catcode`\<=\active \def<{
\fontencoding{T1}\selectfont\symbol{60}\fontencoding{\encodingdefault}}

%%%%%%%%%%%%%%%%%%%%%%%%%%%%%%%%%%%%%%%%%%%%%%%%%%%%%%%%%%%%%%%%%%
%%%%%%%%%%%%%%%%%%%%%%%%%%%%%%%%%%%%%%%%%%%%%%%%%%%%%%%%%%%%%%%%%%
%Begin now!



\begin{document}

\begin{titlepage}
    \begin{center}
        \vspace*{3cm}
            
        \Huge
        \textbf{Machine Learning}
            
        \vspace{1cm}
        \huge
        Homework\hwnumber
            
        \vspace{1.5cm}
        \Large
            
        \textbf{\Information}\\                      % <-- author
        \textbf{\Informatio}\\
        \textbf{\Informati} \\
        \textbf{\Informat} \\
            
        \vfill
        
         \course \ 
            
        \vspace{1cm}
            
        \includegraphics[width=0.4\textwidth]{iitbh.png}
        \\
        
        \Large
        
        \today
            
    \end{center}
\end{titlepage}

%%%%%%%%%%%%%%%%%%%%%%%%%%%%%%%%%%%%%%%%%%%%%%%%%%%%%%%%%%%%%%%%%%
%%%%%%%%%%%%%%%%%%%%%%%%%%%%%%%%%%%%%%%%%%%%%%%%%%%%%%%%%%%%%%%%%%
%Start the assignment now
%%%%%%%%%%%%%%%%%%%%%%%%%%%%%%%%%%%%%%%%%%%%%%%%%%%%%%%%%%%%%%%%%%
%New problem
\newpage
\begin{Problem}
\end{Problem}

\begin{Solution}\\
\textbf{Question 12}\\
Given \\
b = 0, 8, 8, 20\\
t = 0, 1, 3, 4 \\
line equation
\begin{align*}
    C + Dt &= b \\
\end{align*}
So Matrix A and b are
\begin{align*}
A &= \begin{bmatrix}
    1 & 0 \\
    1 & 1 \\
    1 & 3 \\
    1 & 4 \\ 
    \end{bmatrix} 
\\
b &= \begin{bmatrix}
    1\\
    1\\
    1\\
    1 
    \end{bmatrix}
\end{align*}
Now solving  
\begin{align}
A^TAx &= A^Tb \\
x &= (A^TA)^{-1} A^Tb \\
 x &= \begin{bmatrix}
    C\\
    D\\
    \end{bmatrix} 
\nonumber
\end{align}
let's calculate separately
\begin{align*}
A^TA &= \begin{bmatrix}
    1 & 0 \\
    1 & 1 \\
    1 & 3 \\
    1 & 4 \\ 
    \end{bmatrix}
    \begin{bmatrix}
        1 & 1 & 1 & 1 \\
        0 & 1 & 3 & 4
    \end{bmatrix}\\
   &= \begin{bmatrix}
        1+1+1+1 & 1+3+4\\
        1+3+4 & 1+9+16\\
    \end{bmatrix}\\
   &= \begin{bmatrix}
    4 & 8 \\
    8 & 26
\end{bmatrix}\\
\end{align*}
\begin{align*}
(A^TA)^{-1} &= 
    \frac{1}{\mod(A^TA)}
    cofac
\begin{bmatrix}
    4 & 8 \\
    8 & 26
\end{bmatrix}^T\\
 &= \frac{1}{40}\begin{bmatrix}
    26 & -8 \\
    -8 & 4  
 \end{bmatrix}^T\\
 &=
\frac{1}{40}
\begin{bmatrix}
    26 & -8 \\
    -8 & 4  
\end{bmatrix}
\end{align*}
\begin{align*}
A^Tb &= \begin{bmatrix}
    1 & 1 & 1 & 1 \\
    0 & 1 & 3 & 4
\end{bmatrix}
\begin{bmatrix}
    0 \\ 
    8 \\ 
    8 \\ 
    20 \\ 
\end{bmatrix}\\
 &= \begin{bmatrix}
    36 \\
    112
\end{bmatrix}
\end{align*}
from equation (2)
\begin{align}
\nonumber
x &= (A^TA)^{-1} A^Tb \\
\nonumber
 &= \frac{1}{40}
    \begin{bmatrix}
        26 & -8 \\
        -8 & 4  
    \end{bmatrix}
    \begin{bmatrix}
        36 \\
        112
    \end{bmatrix}\\
\nonumber
    &= \frac{1}{40}
    \begin{bmatrix}
         26*36 & (-8)*112 \\ 
        36*(-8) &  4*112 \\ 
    \end{bmatrix} \\
\nonumber
    &= \frac{1}{40}
    \begin{bmatrix}
        40 \\
        160
    \end{bmatrix} \\
    &= 
    \begin{bmatrix}
        1 \\
        4
    \end{bmatrix} \\
\nonumber
\end{align}
Now calculate P \\
we know that 
\begin{align}
\nonumber
    p &= Ax\\
\nonumber
      &=  \begin{bmatrix}
        1 & 0 \\
        1 & 1 \\
        1 & 3 \\
        1 & 4 \\ 
        \end{bmatrix} 
        \begin{bmatrix}
            1 \\
            4
        \end{bmatrix}\\
\nonumber
      &= \begin{bmatrix}
        1+0  \\
        1+4  \\
        1+12 \\
        1+16
      \end{bmatrix}\\
      &= \begin{bmatrix}
        1  \\
        5  \\
        13 \\
        17
      \end{bmatrix} 
\end{align}
So from (4) we can calculate errors as follows
\begin{align*}
    e &= b - p \\
      &= \begin{bmatrix}
            0 \\
            8 \\
            8 \\
            20 \\
      \end{bmatrix} - 
      \begin{bmatrix}
        1 \\
        5 \\
        13 \\
        17 \\
      \end{bmatrix}\\
     &= \begin{bmatrix}
        -1 \\
         3 \\
        -5 \\
         3 \\
     \end{bmatrix} \tag{5}
\end{align*}
from equation (5) we can calculate
\begin{align*}
    E &= e_1^2 + e_2^2 + e_3^2 + e_4^2 \\
      &= (-1)^2 + (3)^2 + (-5)^2 + (3)^2\\
      &= 1 + 9 + 25 + 9\\
      &= 44
\end{align*}
\textbf{So value of minimum squared error is 44 } 
\end{Solution}

%%%%%%%%%%%%%%%%%%%%%%%%%%%%%%%%%%%%%%%%%%%%%%%%%%%%%%%%%%%%%%%%%%

%%%%%%%%%%%%%%%%%%%%%%%%%%%%%%%%%%%%%%%%%%%%%%%%%%%%%%%%%%%%%%%%%%
%New problem
\newpage
\begin{Problem}
\end{Problem}

\begin{Solution}\\
\textbf{Question 13}\\
Given \\
b = 0, 8, 8, 20 \\
t = 0, 1, 3, 4 \\
\\
four equations are(unsolvable)\\
\begin{align*}
    Ax &= b\\
    \begin{bmatrix}
        1 & 0 \\
        1 & 1 \\
        1 & 3 \\
        1 & 4 \\
    \end{bmatrix} 
    \begin{bmatrix}
        C\\
        D
    \end{bmatrix}
     &= \begin{bmatrix}
        0\\
        8\\
        8\\
        20\\
     \end{bmatrix}\\
\end{align*}
\begin{align*}
C &= 0 \tag{1}\\
C + D &= 8 \tag{2}\\  
C + 3D &= 8 \tag{3}\\
C + 4D &= 20 \tag{4} 
\end{align*}
Now we have p = 1, 5, 13, 17 \\
let's find value of x
\begin{align*}
    Ax &= p \\
    A^TAx &= A^Tp\\
    x &= (A^TA)^{-1}A^Tp 
\end{align*}
\\
from Question 12 we know value of $(A^TA)^{-1}$
\begin{align*}
    x &= \frac{1}{40}
    \begin{bmatrix}
       26 & -8 \\ 
       -8 & 4
    \end{bmatrix}
    \begin{bmatrix}
        1 & 1 & 1 & 1  \\
        0 & 1 & 3 & 4 \\
    \end{bmatrix}
    \begin{bmatrix}
        1 \\ 5 \\ 13 \\ 17 \\
    \end{bmatrix}\\
      &= \frac{1}{40}
    \begin{bmatrix}
        26 & 26-8 & 26-24 & 26-32 \\
        -8 & -8+4 & -8+12 & -8+16  \\
    \end{bmatrix}
    \begin{bmatrix}
        1 \\
        5 \\
        13 \\
        17 \\
    \end{bmatrix}\\
    &= \frac{1}{40}
    \begin{bmatrix}
        26 & 18 & 2 & -6 \\
        -8 & -4 & 4 & 8  \\
    \end{bmatrix}
    \begin{bmatrix}
        1 \\
        5 \\
        13 \\
        17 \\
    \end{bmatrix}\\
    &= \frac{1}{40}
    \begin{bmatrix}
        26 + 90 + 26 - 102 \\
        -8 - 20 + 52 + 136 
    \end{bmatrix}\\
    &= \frac{1}{40}
    \begin{bmatrix}
         40 \\ 
         160 \\
    \end{bmatrix}\\
    &= \begin{bmatrix}
        1 \\
        4
    \end{bmatrix}
\end{align*}

\textbf{So the value of x is (1,4) which is same as previous question as we are putting same p from previous question}

\end{Solution}

%%%%%%%%%%%%%%%%%%%%%%%%%%%%%%%%%%%%%%%%%%%%%%%%%%%%%%%%%%%%%%%%%%

%%%%%%%%%%%%%%%%%%%%%%%%%%%%%%%%%%%%%%%%%%%%%%%%%%%%%%%%%%%%%%%%%%
t = (0, 1, 3, 4) \\
b = (0, 8, 8, 20) \\

Now matrix A is (as co-effecient of C are zeros)
\begin{align*}
    A &= \begin{bmatrix}
        0 \\
        1 \\
        3 \\
        4 
    \end{bmatrix} 
\end{align*}
To find
\begin{align*}
    A^TA\hat{x} &=  A^Tb \\
    \hat{x}     &=  (A^TA)^{-1}A^Tb 
\end{align*}
let's calculate\\
\begin{align*}
    A^TA &= \begin{bmatrix}
        0 & 1 & 3 & 4 \\
    \end{bmatrix}
    \begin{bmatrix}
        0 \\
        1 \\
        3 \\
        4 \\
    \end{bmatrix} \quad 
    = [26] \tag{5}
\end{align*}
\begin{align*}
    A^Tb &= \begin{bmatrix}
        0 & 1 & 3 & 4 \\
    \end{bmatrix}
    \begin{bmatrix}
        0 \\
        8 \\
        8 \\
        20 \\
    \end{bmatrix} \quad 
    = [112] \tag{6}
\end{align*}
from equation(5) and (6) 
\begin{align*}
    \hat{x} &= (A^TA)^{-1}A^Tb\\
            &= \frac{1}{26} \times [112]\quad
            = \frac{56}{13} \tag{7}
\end{align*}
\textbf{So the required line equation is }
\begin{align*}
    Dt &= b\\
    \left(\frac{56}{13}\right) t &= b
\end{align*}

%%%%%%%%%%%%%%% Graph Representation %%%%%%%%%%%%%%%%%%%%
\newpage
\textbf{Graphical Representation}\\
\\
\begin{tikzpicture}
    \begin{axis}[
        xlabel={$t$},
        ylabel={$b$},
        xmin=0, xmax=4.5,
        ymin=0, ymax=25,
        axis lines=middle,
        axis line style={-latex},
        xtick={0.01,1,3,4},
        xticklabels={$t_1=0$, $t_2=1$, $t_3=3$, $t_4=4$},
        ytick={0.01,8,20},
        yticklabels={$b_1=0$, $b_2=b_3=8$, $b_4=20$},
        width=15cm,
        height=16cm,
        clip=false,
    ]
    
    % Points b1, b2, b3, b4
    \node[circle,fill,inner sep=1.5pt] at (axis cs:0,0) {};
    \node[circle,fill,inner sep=1.5pt] at (axis cs:1,8) {};
    \node[circle,fill,inner sep=1.5pt] at (axis cs:3,8) {};
    \node[circle,fill,inner sep=1.5pt] at (axis cs:4,20) {};
    
    \draw[dashed] (axis cs:0,0) -- (axis cs:0,0) node[midway, above right, black] {e1};
    \draw[dashed] (axis cs:1,56/13) -- (axis cs:1,8) node[midway, above right, black] {e2};
    \draw[dashed] (axis cs:3,168/13) -- (axis cs:3,8) node[midway, above right, black] {e3};
    \draw[dashed] (axis cs:4,224/13) -- (axis cs:4,20) node[midway, above right, black] {e4};
    
    \node[red,fill=black,circle,inner sep=1.5pt] at (axis cs:1,56/13) {};
    \node[below right] at (axis cs:1,56/13) {$p_2$};
    
    \node[red,fill=black,circle,inner sep=1.5pt] at (axis cs:0,0) {};
    \node[above right] at (axis cs:0,0) {$p_1$};
    
    \node[red,fill=black,circle,inner sep=1.5pt] at (axis cs:3,168/13) {};
    \node[above left] at (axis cs:3,168/13) {$p_3$};
    
    \node[red,fill=black,circle,inner sep=1.5pt] at (axis cs:4,224/13) {};
    \node[below right] at (axis cs:4,224/13) {$p_4$};
    
    % Line b = (56/13)t
    \addplot[color=red,mark=none,domain=0:4.5] {(56/13)*x} node[midway, above, sloped, black] {$b=\frac{56}{13}t$};
    
    \end{axis}
\end{tikzpicture}
%%%%%%%%%%%%%%% Graph Representation %%%%%%%%%%%%%%%%%%%%

\end{Solution}

%%%%%%%%%%%%%%%%%%%%%%%%%%%%%%%%%%%%%%%%%%%%%%%%%%%%%%%%%%%%%%%%%%

%%%%%%%%%%%%%%%%%%%%%%%%%%%%%%%%%%%%%%%%%%%%%%%%%%%%%%%%%%%%%%%%%%
%New problem
\newpage
\begin{Problem}
\end{Problem}

\begin{Solution}\\
\textbf{Question 19}\\
Given vectors \\
b = (0, 8, 8, 20) \\
a = (0, 1, 3, 4) \\ 

To find projection of b along a :-
\begin{align*}
    a.b &= |a||b|cos\theta \tag{dot product} \\
    cos\theta &= \frac{a.b}{|a||b|} \tag{1}\\
\end{align*}
component of b along a
\begin{align*}
    &=(|b|cos\theta)\hat{a} \\
    &= |b| \left(\frac{a.b}{|a||b|} \right)\left(\frac{a}{|a|}\right) \\
    &= \frac{a.b}{|a||a|} a \tag{2}
\end{align*}
we can also write equation(2) as 
\begin{align*}
    &= \left(\frac{a^Tb}{a^Ta}\right)a \tag{3}\\
\end{align*}
from above equations 
\begin{align*}
    p &= \hat{x}a \\
    \therefore \hat{x} &= \frac{a^Tb}{a^Ta} \\
\end{align*}
let's calculate $\hat{x}$ (taking values of $a^Tb$ and $a^Ta$ from question 18 [equation(5) and (6)] )
\begin{align*}
    \hat{x} &= \frac{a^Tb}{a^Ta} \quad
        = \frac{112}{26} \quad
        = \frac{56}{13}
\end{align*}
Calculating p
\begin{align*}
    p &= \hat{x} a  \\
      &= \frac{56}{13} 
      \begin{bmatrix}
        0 \\
        1 \\
        3 \\
        4 \\
      \end{bmatrix}  \\
\end{align*}
from Problem 16 and Problem 18

(C,D) = (9,56/13)\\
from Problem 11-14

(C,D) = (1,4)\\
\\

\textbf{These are not same because vector $(1,1,1,1)$ and $(0,1,3,4)$ are not\newline perpendicular }

\end{Solution}

%%%%%%%%%%%%%%%%%%%%%%%%%%%%%%%%%%%%%%%%%%%%%%%%%%%%%%%%%%%%%%%%%%

%%%%%%%%%%%%%%%%%%%%%%%%%%%%%%%%%%%%%%%%%%%%%%%%%%%%%%%%%%%%%%%%%%
%New problem
\newpage
\begin{Problem}
\end{Problem}

\begin{Solution}\\
\textbf{Problem 20}\\
Given\\
equation of Parabola
\begin{align*}
    b &= C + Dt + Et^2 
\end{align*}
vectors are\\
b = (0, 8, 8, 20)\\
t = (0, 1, 3, 4)\\
So out unsolvable equation will be
\begin{align*}
    Ax &= b  \\
    \begin{bmatrix}
        1 & 0 & 0  \\
        1 & 1 & 1  \\
        1 & 3 & 9  \\
        1 & 4 & 16 \\
    \end{bmatrix}
    \begin{bmatrix}
        C \\
        D \\
        E \\
    \end{bmatrix}
    &= 
    \begin{bmatrix}
        0\\
        8\\
        8\\
        20\\
    \end{bmatrix}
\end{align*}
Now solve for equation
\begin{align*}
    A^TAx &= A^Tb 
\end{align*}
Calculating both separately
\begin{align*}
    A^TA &= \begin{bmatrix}
        1 & 1 & 1 & 1  \\
        0 & 1 & 3 & 4  \\
        0 & 1 & 9 & 16  \\
    \end{bmatrix} 
    \begin{bmatrix}
        1 & 0 & 0  \\
        1 & 1 & 1  \\
        1 & 3 & 9  \\
        1 & 4 & 16 \\
    \end{bmatrix} \\
    &= \begin{bmatrix}
        1+1+1+1 & 0+1+3+4 & 0+1+9+16 \\
        0+1+3+4 & 0+1+9+16 & 0+1+27+64 \\
        0+1+9+16 & 0+1+27+64 & 0+1+81+256 \\
    \end{bmatrix} \\
    &= \begin{bmatrix}
        4 & 8 & 26  \\
        8 & 26 & 92  \\
        26 & 92 & 338  \\
    \end{bmatrix} \\
    A^Tb &= \begin{bmatrix}
        1 & 1 & 1 & 1  \\
        0 & 1 & 3 & 4  \\
        0 & 1 & 9 & 16  \\
    \end{bmatrix} 
    \begin{bmatrix}
        0  \\
        8  \\
        8  \\
        20  \\
    \end{bmatrix} \\
    &= \begin{bmatrix}
        0+8+8+20 \\
        0+8+32+80  \\ 
        0+8+72+320 \\
    \end{bmatrix} \\
    &= \begin{bmatrix}
        36\\
        112\\
        400\\
    \end{bmatrix}
\end{align*}
\textbf{So our Normal equation will be}
\begin{align*}
    A^TAx &= A^Tb \\
    \begin{bmatrix}
        4 & 8 & 26  \\
        8 & 26 & 92  \\
        26 & 92 & 338  \\
    \end{bmatrix}
    \begin{bmatrix}
        C \\
        D \\
        E \\
    \end{bmatrix} &=
    \begin{bmatrix}
        36\\
        112\\
        400\\
    \end{bmatrix} \\
\end{align*}
\textbf{Figure II.3b will not change as it's just projection in 4 dimension vector $[R^4]$ same as given figure }
 
\end{Solution}

%%%%%%%%%%%%%%%%%%%%%%%%%%%%%%%%%%%%%%%%%%%%%%%%%%%%%%%%%%%%%%%%%%

%%%%%%%%%%%%%%%%%%%%%%%%%%%%%%%%%%%%%%%%%%%%%%%%%%%%%%%%%%%%%%%%%%
%New problem
\newpage
\begin{Problem}
\end{Problem}

\begin{Solution}\\
\textbf{Question 21}\\
Given\\
Cubic equation
\begin{align*}
    b &= C + Dt + Et^2 + Ft^3
\end{align*}
Writing in form of $Ax = b$
\begin{align*}
    \begin{bmatrix}
        1 & 0 & 0 & 0 \\
        1 & 1 & 1 & 1 \\
        1 & 3 & 9 & 27 \\
        1 & 4 & 16 & 64 
    \end{bmatrix}
    \begin{bmatrix}
        C\\
        D\\
        E\\
        F\\
    \end{bmatrix}
    &= \begin{bmatrix}
        0\\
        8\\
        8\\
        20
    \end{bmatrix}
\end{align*}
We have system of linear equation as follows 
\begin{align*}
    C &= 0 \tag{1} \\
    C + D + E +F &= 8 \tag{2} \\
    C + 3D + 9E +27E &= 8 \tag{3} \\
    C + 4D + 16E + 64 F&= 20 \tag{4} 
\end{align*}
\textbf{solving system of linear equations by elimination method}

Step 1: Subtract the first row from the second, third, and fourth rows:
\[
\begin{bmatrix}
    1 & 0 & 0 & 0 \\
    0 & 1 & 1 & 1 \\
    0 & 3 & 9 & 27 \\
    0 & 4 & 16 & 64 \\
\end{bmatrix}
\begin{bmatrix}
    C\\
    D\\
    E\\
    F\\
\end{bmatrix}
=
\begin{bmatrix}
    0\\
    8\\
    8\\
    20\\
\end{bmatrix}
\]

Step 2: Subtract 3 times the second row from the third row and 4 times the second row from the fourth row:
\[
\begin{bmatrix}
    1 & 0 & 0 & 0 \\
    0 & 1 & 1 & 1 \\
    0 & 0 & 6 & 24 \\
    0 & 0 & 12 & 60 \\
\end{bmatrix}
\begin{bmatrix}
    C\\
    D\\
    E\\
    F\\
\end{bmatrix}
=
\begin{bmatrix}
    0\\
    8\\
    -16\\
    -12\\
\end{bmatrix}
\]

Step 3: Divide the third row by 6:
\[
\begin{bmatrix}
    1 & 0 & 0 & 0 \\
    0 & 1 & 1 & 1 \\
    0 & 0 & 1 & 4 \\
    0 & 0 & 12 & 60 \\
\end{bmatrix}
\begin{bmatrix}
    C\\
    D\\
    E\\
    F\\
\end{bmatrix}
=
\begin{bmatrix}
    0\\
    8\\
    \frac{-16}{6}\\
    -12\\
\end{bmatrix}
\]

Step 4: Devide by 12 to the fourth row:
\[
\begin{bmatrix}
    1 & 0 & 0 & 0 \\
    0 & 1 & 1 & 1 \\
    0 & 0 & 1 & 4 \\
    0 & 0 & 1 & 5 \\
\end{bmatrix}
\begin{bmatrix}
    C\\
    D\\
    E\\
    F\\
\end{bmatrix}
=
\begin{bmatrix}
    0\\
    8\\
    \frac{-8}{3}\\
    -1\\
\end{bmatrix}
\]

Step 5: Subtract third row from fourth row 
\[
\begin{bmatrix}
    1 & 0 & 0 & 0 \\
    0 & 1 & 1 & 1 \\
    0 & 0 & 1 & 4 \\
    0 & 0 & 0 & 1 \\
\end{bmatrix}
\begin{bmatrix}
    C\\
    D\\
    E\\
    F\\
\end{bmatrix}
=
\begin{bmatrix}
    0\\
    8\\
    \frac{-8}{3}\\
    \frac{5}{3}\\
\end{bmatrix}
\]

Step 6: Subtract the four times of fourth row from the third row :
\[
\begin{bmatrix}
    1 & 0 & 0 & 0 \\
    0 & 1 & 1 & 1 \\
    0 & 0 & 1 & 0 \\
    0 & 0 & 0 & 1 \\
\end{bmatrix}
\begin{bmatrix}
    C\\
    D\\
    E\\
    F\\
\end{bmatrix}
=
\begin{bmatrix}
    0\\
    8\\
    \frac{-28}{3}\\
    \frac{5}{3}\\
\end{bmatrix}
\]

Step 7: Subtract the fourth row from the second row and Subtract the third row from second row :
\[
\begin{bmatrix}
    1 & 0 & 0 & 0 \\
    0 & 1 & 0 & 0 \\
    0 & 0 & 1 & 0 \\
    0 & 0 & 0 & 1 \\
\end{bmatrix}
\begin{bmatrix}
    C\\
    D\\
    E\\
    F\\
\end{bmatrix}
=
\begin{bmatrix}
    0\\
    \frac{47}{3}\\
    \frac{-8}{3}\\
    \frac{5}{3}\\
\end{bmatrix}
\quad
= \frac{1}{3}
\begin{bmatrix}
0 \\
47 \\
-28 \\
5 \\
\end{bmatrix}
\]
\textbf{Solution is}
\begin{align*}
    \begin{bmatrix}
        C\\
        D\\
        E\\
        F\\
    \end{bmatrix} &= 
    \frac{1}{3}
    \begin{bmatrix}
        0\\
        47\\
        -28\\
        5\\
    \end{bmatrix} \\
\end{align*}

\textbf{As we are getting solution to the given equations that means all 
points lies on the Cubic part, so error(e) is zero and p in nothing but b}\\

\textbf{e =(0,0,0,0), p=(0,8,8,20)} 
\end{Solution}

%%%%%%%%%%%%%%%%%%%%%%%%%%%%%%%%%%%%%%%%%%%%%%%%%%%%%%%%%%%%%%%%%%

%%%%%%%%%%%%%%%%%%%%%%%%%%%%%%%%%%%%%%%%%%%%%%%%%%%%%%%%%%%%%%%%%%
%New problem
\newpage
\begin{Problem}
\end{Problem}

\begin{Solution}\\
\textbf{Question 22}\\
Given\\ 
\textbf{Part a]}
\begin{align*}
\bar{t} &= 2 \\
\bar{b} &= 9  
\end{align*}
To verify
\begin{align*}
   C + D\bar{t} &= \bar{b} \tag{1}
\end{align*}
from Question 12 we can take 
\begin{align*}
    \begin{bmatrix}
        C\\
        D
    \end{bmatrix} &= 
    \begin{bmatrix}
        1\\
        4
    \end{bmatrix}
\end{align*}
\begin{align*}
    1 + 4\times\bar{t} &= \bar{b}\\
    1 + 4\times(2) &= (9)\\
    1 + 8 &= 9\\
    9 &= 9 \tag{verified}
\end{align*}
\textbf{Explanation}\\
we know that for best line 
\begin{align*}
    \sum e_i =0 
\end{align*}
also 
\begin{align*}
   e_i &= b_i - p_i \\
   p_i &= b_i - e_i 
   \sum p_i &= \sum b_i - \sum e_i`
   \sum p_i &= \sum b_i
\end{align*}
best line equation is 
\begin{align*}
    \begin{bmatrix}
        1 & t_1\\
        1 & t_2\\
        1 & t_3\\
        1 & t_4
    \end{bmatrix}
    \begin{bmatrix}
        C\\
        D
    \end{bmatrix}
    &= 
    \begin{bmatrix}
        p_1\\
        p_2\\
        p_3\\
        p_4
    \end{bmatrix}
\end{align*}
by adding all equations we get
\begin{align*}
    4C + D\sum t)i &= \sum p_i  \\
    4C + D\sum t)i &= \sum b_i  \\
    C + D\frac{\sum t_i}{4} &= \frac{\sum b_i}{4}  \\
    C + D\bar{t} = \bar{b}
\end{align*}
\textbf{So we can conclude that point $(\bar{t},\bar{b})$ lie on best fit line}
\\
\\
\textbf{Part b]}\\
Given equation 
\begin{align*}
A^TA\hat{x} = A^Tb\\
\end{align*}
where 
\begin{align*}
    A &= 
    \begin{bmatrix}
        1 & t_1 \\      
        1 & t_2 \\      
        1 & t_3 \\      
        1 & t_4 \\      
    \end{bmatrix}\\
    A^TA &= 
    \begin{bmatrix}
        1 & 1 & 1 & 1 \\
        t_1 & t_2 & t_3 &  t_4 \\
    \end{bmatrix}
    \begin{bmatrix}
        1 & t_1 \\      
        1 & t_2 \\      
        1 & t_3 \\      
        1 & t_4 \\
    \end{bmatrix}\quad  
    \begin{bmatrix}
        1 + 1 + 1 + 1 &  t_1+t_2+t_3+t_4 \\
        t_1+t_2+t_3+t_4 & t_1^{2}+t_2^{2}+t_3^{2}+t_4^{2}
    \end{bmatrix}\\
    &= 
    \begin{bmatrix}
        m  & \sum t_i \\
        \sum t_i & \sum t_i^2
    \end{bmatrix}\\
    A^Tb &= 
    \begin{bmatrix}
        1 & 1 & 1 & 1 \\
        t_1 & t_2 & t_3 &  t_4 
    \end{bmatrix}
    \begin{bmatrix}
        b_1 \\
        b_2 \\
        b_3 \\
        b_4 \\
    \end{bmatrix}\quad
    = \begin{bmatrix}
        b_1 + b_2 + b_3 + b_4 \\
        b_1t_1 + b_2t_2 + b_3t_3 + b_4
    \end{bmatrix}\\
   &= \begin{bmatrix}
       \sum b_i \\
       \sum b_it_i
   \end{bmatrix} 
\end{align*}
Combining all we get ==>
\begin{align*}
    A^TA\hat{x} = A^Tb\\
    \begin{bmatrix}
        m  & \sum t_i \\
        \sum t_i & \sum t_i^2
    \end{bmatrix}
    \begin{bmatrix}
        C\\
        D
    \end{bmatrix}
    &=
    \begin{bmatrix}
       \sum b_i \\
       \sum b_it_i
   \end{bmatrix} 
\end{align*}
\textbf{First equation is }
\begin{align*}
    mC + D\sum t_i &= \sum b_i\\
    C + D \frac{\sum t_i}{m} &= \frac{\sum b_i}{m}\\
    C + D \bar{t} &= \bar {b} 
\end{align*}
\end{Solution}

%%%%%%%%%%%%%%%%%%%%%%%%%%%%%%%%%%%%%%%%%%%%%%%%%%%%%%%%%%%%%%%%%%

%Complete the assignment now
\end{document}

%%%%%%%%%%%%%%%%%%%%%%%%%%%%%%%%%%%%%%%%%%%%%%%%%%%%%%%%%%%%%%%%%%
%%%%%%%%%%%%%%%%%%%%%%%%%%%%%%%%%%%%%%%%%%%%%%%%%%%%%%%%%%%%%%%%%%
